\documentclass[11pt]{amsart}
\usepackage{geometry}                % See geometry.pdf to learn the layout options. There are lots.
\geometry{letterpaper}                   % ... or a4paper or a5paper or ... 
\usepackage{graphicx}
\usepackage{amssymb}
\usepackage{dsfont}
\usepackage{epstopdf}
\usepackage{array}
\usepackage{booktabs}%

\DeclareGraphicsRule{.tif}{png}{.png}{`convert #1 `dirname #1`/`basename #1 .tif`.png}
\author {F. Benmansour}
\title{\textit{Constructing metrics from the flux filters}}
\begin{document}
\maketitle

\section{Definitions and notations}
\begin{itemize}
\item The normalised oriented flux matrix of an image $I$ is defined as follows [Law08] :
\begin{equation}\label{eq:OF}
\forall x\in\mathbb{R}^n,\forall r >0, ~~\mathcal{F}(x, r) = (I* \partial_{i, j} g_{\sigma} *\mathds{1}_r)(x)
\end{equation}
where, $g_\sigma$ is the Gausian function with standard deviation $\sigma$, and $\mathds{1}_r = 1/ \mathcal{N}(r)$ if $\|x\| < r$ and 0 otherwise, where $\mathcal{N}(r)$ is the flux normalisation term. Typically, for $n = 3$,  $\mathcal{N}(r) = 4\pi r^2$ and for $n = 2$, $\mathcal{N}(r) = 2\pi r$.
\item The normalised anti-symmetry flux vector is defined as follows [Law10]:
\begin{equation}\label{eq:AF}
\forall x\in\mathbb{R}^n,\forall r >0, ~~\mathcal{AS}(x, r) = (I* \partial_{i} g_{\sigma} *\Delta_r)(x)
\end{equation}
where $\Delta_r(x) = 1/ \mathcal{N}(r)$ if $\|x\| = r$ and 0 otherwise.
\end{itemize}

\section{Isotropic Metrics}
From now on, we assume that we are interested in extracting tubular structures in an image $I$ that are brighter than the background.

At a scale space location $(x, r)$, the matrix $\mathcal{F}(x, r)$. Let $\lambda_1\leq \lambda_2\leq\dots\leq\lambda_n$ be its eigenvalues and $(v_1, v_2, .\dots, v_n)$ be the associated orthonormal basis. We have 
\begin{equation}\label{eq:diago}
\mathcal{F}(x, r) = \sum_{i=1}^n \lambda_i v_i v_i^T
\end{equation}

The cross section trace measure as presented in [Law08] is 
\begin{equation}\label{eq:CrossSectionTrace}
\mathcal{CT}(x, r) = \sum_{i=1}^{n-1} \lambda_i 
\end{equation}
One can use the scale space version of the cross section trace measure, or a space ply one obtained by maximising it along the scale dimension. That is,
\begin{equation}\label{eq:CrossSectionTraceSpace}
\mathcal{CT}(x) = \max_{r} \mathcal{CT}(x, r).
\end{equation}

Let the associated oriented flux matrix be:
\begin{equation}\label{eq:OFSpace}
\mathcal{F}^*(x) = \mathcal{F}(x, r^*),
\end{equation}
where $r^* = \displaystyle\text{argmax}_{r} ~\mathcal{CT}(x, r)$.


\section{Riemannian Metrics}
One can construct an anisotropy metric using the previous notations, and as presented in [Benmansour11], 
\begin{equation}\label{eq:RiemannianMetric1}
\mathcal{R}(x, r) = \sum_{i=1}^n \exp\left[-\alpha\sum_{j\neq i}\lambda_j\right]v_i v_i^T,
\end{equation}
with $\alpha > 0$ is a parameter that is related to the anisotropy ratio as explained later. One can motivate the choice given by equation (\ref{eq:RiemannianMetric1}), by the fact that for a point on the centreline of a tubular structure $v_1$ is an estimate of the direction on which we would like to have a high anisotropy.

The parameter $\alpha$ relates to the anisotropy ratio as follows
\begin{equation}\label{eq:RiemannianAnisoRatio}
\kappa(\mathcal{R}(x, r)) =  \exp\left[\frac{\alpha}{2}\left(\sum_{j\neq 1}\lambda_j-\sum_{j\neq n}\lambda_j  \right)\right] = \exp\left[\frac{\alpha}{2}\left(\lambda_n - \lambda_1 \right)\right],
\end{equation}
By setting the maximal anisotropy ratio to $\kappa_{\max}$, 
\begin{equation}\label{eq:alpha}
\alpha =  \frac{2 \log(\kappa_{\max})}{\max( \lambda_n - \lambda_1)}
\end{equation}

When operating on the scale space domain, the constructed Riemannian metric is of dimension $(n+1)$, which is an augmentation of $\mathcal{R}(x, r)$ with a \lq\lq scale speed\rq\rq $\mathcal{P}(x, r)$. We want this speed to be always higher than the spatial speed so that our tubular tracked first adapt to scale then it adapts to location.
Say that the user decides on the ratio of the scale speed compared to the best spatial speed, we call it $\eta$, then one can take
\begin{equation}\label{eq:alpha}
\mathcal{P}(x, r)=  \eta\exp\left[-\alpha \sum_{j\neq n} \lambda_i \right].
\end{equation}

TODO, I still believe that the solution proposed in the IJCV paper is better...but now let's code.

\section{Finslerian Metrics}
TODO, figure out how to combine the pervious stuff with the anti-symmetry vector.
\end{document}